\begin{abstract}
	A predictive transport computational model that solves a recently postulated form of the plasma fluid equations \cite{Stacey2017} that conserves particles, momentum, and energy will be implemented. The transport equations set will be extended and modified as necessary to provide reasonable agreement with experimental data collected from DIII-D, a research tokamak owned by General Atomics. Thus, a validation exercise comparing against a variety of H-mode, L-Mode and RMP data sets is anticipated in the development of the computational model. The transport equations were proposed to capture the radial distribution resulting from a neutral beam source, which inserts particles, momentum, and energy, ion orbit loss, which removes particles and their associated momentum and energy in a non-diffusive manner, and the equilibrium field conditions of toroidal and poloidal rotation velocities, plasma pressure, density, and temperature. The equations as presented are first order, coupled, non-linear equations, resulting from the long range Lorentz forces $\left(\vxb\right)$ and electric field $\left(\vec{E}\right)$. Therefore, iterative computational techniques will be employed. The primary solver that will be explored is a variant of the Newton-Raphson multivariate solver known as Broyden’s method. A pure Newton-Raphson solver evaluates the Jacobian analytically, but the nature of our equations will require a numerical approximation of the Jacobian. In the course of the analysis if other iterative techniques arise, we will consider implementing them for comparison.
\end{abstract}
